% Copyright (c) 2014,2016 Casper Ti. Vector
% Public domain.

\chapter{序言}
\section{研究背景}
当今,人们愈加重视对疾病的预防和监测,而体外诊断是近些年来新兴并受到欢迎的一种有效监测手段和方法。它通过对某些人体样本(如体液、血液、组织液等)进行检测,判断其中是否存在疾病的生化标志物,从而及时的对疾病进行有效地掌控和治疗。体外诊断一种不可或缺的有机组成是血液的检测与分析,然而伴随着现代化进程的不断推进,人们不断增高的各种发病率使得进行血液样本分析的工作量剧增。与此同时,临床技术的发展也使得血液样本的分析从传统的手工操作到半自动化分析,进一步发展到现在的全自动检测。当今,最广泛使用的全自动血液测试和分析仪器是自动生化分析仪。它可用于临床生物化学的常规分析,以及对分泌激素,排泄物,脑脊液成分,有毒溶液和电解质的检测。这些测试为临床医学和科学研究实验提供了极大的便利。由于其强大的功能,全自动生化分析仪近年来已成为临床实验室分析中最常用的测试仪器。
\section{自动生化分析仪国内外研究现状}
自动生化分析仪是一种集多功能与一体的医学实验室仪器,它能够用于定量测量和分析人体血液,尿液和其他体液各种生物标志物和化学元素,如:微量元素和其他电解质,以及激素和微蛋白,并对肝功能,肾功能,心脏功能进行监测。人体液体生化指标具有重要的参考价值,它为临床医生在临床疾病诊断和治疗提供了原始数据。从全自动生化仪的 提出到现在大约半个世纪,生化分析仪得到了广泛的应用和迅速发展。目前在我国,生化分析仪的发展落后于先进国家。

自动生化分析仪在国外的研究起步于20世纪50年代,第一台自动生化分析仪诞生于瑞士。随着应用市场不断扩大,自动生化分析仪的研究迅速发展,其技术也日臻成熟,国外各龙头医疗器械企业都研制出自己的产品,比如瑞士的澳斯邦及帝肯、美国的强生、德国的西门子及罗氏、日本的奥林巴斯公司等等\supercite{bib2},这些国外科技势力发展迅速,这些国际龙头企业不断对自动生化分析仪硬件和软件系统进行更新,增加和完善了分析仪的许多功能,使检测仪器趋于完美。其产品具有加样速度快、位置精度高、抗干扰性强等优点。

我国对自动生化分析仪的研究开展比较晚,其起始于1972年引进美国泰克尼康公司流动式生化分析仪,在此基础上,我国研制出了第一台自动生化分析仪。上世纪八九十年代是我国自动生化分析仪发展的黄金时期,自动生化分析水平与国际先进水平之间的差距越来越小,但大多数的研发机器都处在样机阶段,不能进入应用市场。2003年,深圳奥迈Ominilab BS-300的面市才使得我国拥有了能商业化的全自动生化分析仪\supercite{bib1}。

在自动加样那个系统的相关研究中,文献\parencite{bib2}在完成机械机构、运动部件的设计和嵌入式控制的基础上,重点对加样系统关键的机械部件利用有限元软件进行了力学分析和功能仿真研究;文献\parencite{bib3}在搭建实验平台的基础上,重点研究了加样精度的控制算法和微量加样器的设计;文献\parencite{bib8}在完成自动加样系统的机械设计的基础上,通过对自动加样机的过载故障的研究,重点关注在步进电机细分驱动和调速控制上,并通过对正常组件和故障组件测量来验证效果;文献\parencite{bib9}在完成加样机械臂结构和尺寸的设计之后,重点建立相关数学模型进行了机械臂的运动学和动力学分析。这些研究成果均对自动加样系统的某一部分进行了重点研究,完善了加样系统设计方式,相比于半自动加样系统在加样效率有了一定的提高。然而,这些文献没有考虑控制成本及操作友好性,加样的精度和效率还有待提高。此外,设计的加样系统对小型研究团体没有针对性,存在体积大、部件价格贵、控制成本高等问题。
\section{论文研究意义及内容}
作为21世纪新的科研领域,自动生化分析仪的开发和应用受到医学,海关防疫,生物工程和生命科学等科技部门的高度重视。同时,需要处理的检测样本量指数式增加增加,使得实验人员面临的劳动强度十分严峻。全自动加样系统作为自动生化分析仪的核心部分,已成为朝着高精度和高质量发展的重点研究目标。

一般地,待测样本的加样量过小过大都会引起测定误差,进而影响试液最后的测定精度。一方面,我国的自动生化分析仪在加样精度指标上与国际先进水平样品加样的精度有很大的差距,国外领先水平的仪器加样最小分辨率达到了0.001$ml$数量级,而我国加样精度仍停留在0.05$ml$数量级。另一方面,国产自动生化分析仪存在灵敏度低、样本消耗多、控制系统精度差等诸多问题。因此,进行自动生化分析仪的自动加样系统的研究和设计的需求显得十分突出。而加快对自动加样系统的研究也有利于提升我国检测仪器的模块化发展,提高检测仪器系统部件的互换性,缩小与国际先进水平的差距,同时为人们健康提供有力的监测保障。此外,全自动加样系统的好处不仅包括减少医检实验人员手工劳动和操作危险试液所涉及的风险,还可以提高数据完整性,降低加样误差,提高准确性,加快分析过程,减少试液的耗用量从而降低成本,避免样品污染和人为错误。

当代检测技术对加样的准确性、重复性稳定性等要求越来越高。因此开展对全自动生化分析仪加样控制系统的研究,有利于提高国内全自动生化分析仪的设计生产水平以及测试准确性,推动国内样品检测技术、改善民众医疗服务条件。针对目前这一现状,为保证在保证一定加样精度的同时,提高加样效率、降低控制成本。本文设计了一种应用于自动生化分析仪的全自动加样系统,该系统的设计主要设计机电一体化技术、自动控制技术、传感器技术等。本文的研究在利用比较法、文献法、分析法、测试法基础上将以理论研究、软件、硬件分析与设计相结合的方式进行。将理论分析研究具体应用到控制方案中。具体如下:

第一章 ,查阅探究相关文献,论述全自动生化分析仪国内外发展状况,分析自动加样系统工作原理及相关技术,这是整个自动化加样系统设计的基础。

第二章,根据功能需求,设计自动加样系统的动作流程并确定系统相关的部件组成。

第三章,主要对加样机械臂的机械系统的设计,确定机械传动系统的驱动方式、传动方式和电机选型等,进行机械臂结构的设计。

第四章,据加样系统对定位控制要求,制定自动加样系统总体控制方案以及机械运动部分控制方案,包括根据励磁方式对步进电机选型,步进电机细分控制的分析,步进电机驱动器的分析与选型等。

第五章,根据控制方案和相关系统功能部件,进行硬件设计与选型,如下位机控制器的选取,驱动器的选取,绘制相关电路图。


% vim:ts=4:sw=4
